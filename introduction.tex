\section{Introduction}\label{sect:introduction}

In the past decade, the video game industry has seen a dramatic increase in its market share and in the player experience and technical complexity of its products. These changes have produced several new needs for game developers and players. For instance, the raising of rich and persistent online and offline games such as \emph{Massive Multiplayer Online Role-Playing Games} (MMORPG), large \emph{Role-Play Games} (RPG), \emph{Social Games} and \emph{Augmented Reality Games} (ARG), is creating longterm data on players within games and developers are pressed to develop content that provides longterm engagement. 

A new type of \emph{Non-Player Characters} (NPCs), that is capable to interact autonomously with the player over a long time period and multiple game sessions, may provide, in the near future, an innovative and advantageous solution for these developers needs. In fact, this \emph{lifelong} NPCs can be able to exploit the information coming over multiple game sessions, from the interaction with the players and other NPCs, in order to offers more realistic and rich behaviors, such as to increase the player engagement with the non-player population of the game.

Unfortunately, this agent's ``long-term memory'' capability is relatively unexplored in video-games domains. As a consequence, modern persistent NPCs (e.g., town inhabitants or quest-givers) use the same artificial intelligence techniques designed for characters which spawn and get killed after a short time.

This approach clearly shows its drawbacks when the game requires frequent interactions among the player and other persistent NPCs. Even triple-A games like \emph{The Elder Scroll V: Skyrim}\footnote{\url{http://en.wikipedia.org/wiki/The_Elder_Scrolls_V:_Skyrim}} and \emph{World of Warcraft}\footnote{\url{http://en.wikipedia.org/wiki/World_of_Warcraft}} suffer of this lack: the player \emph{feeling of realism} is constantly  weakened by a number of NPC's ``silly'' behaviors such as, for example, town guards which do not remember the player and treat him like a stranger even after several game hours, or the instantaneous propagation of information (e.g., a crime committed) between all the NPCs of the kingdom. The result is that the persistent NPCs artificial intelligence is one of the players community main concern about this type of games.

In this research proposal, we manage to tackle the challenge to develop a lifelong agent in a video game application. We believe that the introduction of practical lifelong agents can produce benefits to both video game industry and also other non-game aspects of every day life.