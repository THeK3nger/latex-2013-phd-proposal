\section{Results and Impact}

As stated previously, the development of \emph{lifelong NPCs} can considerably improve the player experience and the developers possibilities in a number of recent rising game genres such as \emph{Massive Multiplayer Online Role-Playing Games} (MMORPG) and \emph{Social Games}. Moreover, they can also help to improve challenge and realism in old-school genres like \emph{Tycoon} and \emph{Simulation} games.

With regard to online persistent world (typical of MMORPG), lifelong agents can be used to fill many gaps, such as, for instance:

\begin{enumerate}
\item Lifelong NPCs can remember the previous interaction with the players. As a consequence, players and lifelong NPC can interact each other using more believable and natural dialogs. For example, it will be possible to produce complex multi-agent phrases such as ``Hi! Where is your friend Foo? I have not seen him for some time.'' or  ``Yeah, I saw Bar! He has just passed through here!''
%
\item Lifelong NPCs can remember and elaborate past events. This open the possibility to develop a group of persistent agents (e.g., town inhabitants) able to remember their own life in the environment (e.g., the town) and automatically building their own background story in the game context. Thus, lifelong NPCs can discuss about or communicate to other players past milestone events (for instance, great battles or particular monster attacks), cite notable players which they have encountered and, in practice, build their own mythology on the basis of the players past interaction.
%
\item Lifelong NPCs can continuously learn during the game. They can classify player behavior or ability, learn to avoid particular players or locations. This can be used also as heuristic and refining tool for planning purpose.
%
\item Lifelong NPC may be used to improve player engagement and social interaction with computer companions (or rival). This type of NPCs follow the player throughout the game and share the same player experience. However, modern games can not handle this situation in a proper way. As a consequence, companion NPCs always seems to be puppets instead of intelligent agents. We believe that lifelong NPCs can dramatically increase this character reliability.
\end{enumerate}

Another important application of lifelong agents in game concerns persistent data among multiple games. Many modern games usually offers the possibility to import previous \emph{save file} to new titles in order to provide a \emph{sense of continuity} in the player experience. For instance, the \emph{Mass Effect} series propagate the information about some important fixed decisions in order to enforce the plot continuity between all the trilogy. The study of lifelong agents can significantly improve this sought-after feature in the following ways:

\begin{enumerate}
\item Each lifelong NPC can easily transfer its past experience in the new game. Since the NPC behavior and knowledge is based on a standardized internal database rather than on hardcoded game variables, it will be more easy to put this knowledge base into another similar game.
%
\item A disembodied NPC can be used to store player information about the whole plot of the game. Then, it can be transferred like any other game NPCs in order to propagate to the new game the whole player experience.
\end{enumerate}

Lifelong agents may be also used to handle the emerging problem of NPC's interaction with \emph{user generated contents}. Combined with a \emph{smart-objects-like} infrastructure, it will be possible to automatically learn unknown object functionality and plan with them \cite{abaci2005planning}. [TODO]

At the same time, lifelong agents in game can be also used as disembodied game character (\emph{game-masters}, \emph{drama managers}, player assistants/commanders and so on) which can procedurally generate new game contents and tasks on the basis of their knowledge of the player previous experience in the game. Actually, games which use procedurally generated contents (mostly \emph{rogue-like} and \emph{sandbox} games) use random seeds to always create new experience to the player. Using lifelong disembodied character could lead to a new type of \emph{adaptive games} not yet explored nowadays where the contents are tailor-made around the single player's tastes and abilities.

Finally, another important application for lifelong NPCs is the \emph{Augmented Reality} scenario. Augmented Realty is an increasingly important field both for gaming than for emerging services. In this setting, the real world is the environment in which virtual agents live and the player (or user) can interact with them using smartphones, tablets and any other suitable device. Over that for games, Augmented Reality and lifelong agents can be used as tools for \emph{gamification}\footnote{The use of game thinking and game mechanics in a non-game context to engage users and solve problems} services \cite{huotari2012defining}. Even in this case, the intrinsic persistent nature of the environment (i.e., the real world) guarantees that lifelong agents can improve the overall quality of the game or application.

In summary, lifelong NPCs are a relatively new, unexplored and promising research field. The introduction of lifelong NPCs in game can provide many practical benefits and help to deal with problems that can hardly be solved by traditional methods. Moreover, the preliminary study of generic lifelong agents can contribute to many other academic fields such as knowledge representation and reasoning, planning, robotics, and so on.

